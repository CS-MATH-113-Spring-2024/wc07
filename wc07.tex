\documentclass[a4paper]{exam}

\usepackage{amsmath}
\usepackage[a4paper]{geometry}

\usepackage{draftwatermark}
\SetWatermarkText{Sample Solution}
\SetWatermarkScale{3}
\printanswers


\header{CS/MATH 113}{WC07: Proofs}{Spring 2024}
\footer{}{Page \thepage\ of \numpages}{}
\runningheadrule
\runningfootrule

\printanswers

\qformat{{\large\bf \thequestion. \thequestiontitle}\hfill}
\boxedpoints

\title{Weekly Challenge 07: Proofs}
\author{CS/MATH 113 Discrete Mathematics}
\date{Spring 2024}

\begin{document}
\maketitle

\begin{questions}

\titledquestion{Perfect Universe}[5]
  We consider a \textit{perfect universe}, $U$, that is inhabited by distinct elements. Any two elements, $a$ and $b$, can be \textit{combined} to yield a unique element $c$, denoted as $ab=c$. The following properties holds in $U$.
  \begin{itemize}
  \item Associativity: $\forall a\forall b\forall c [(ab)c = a(bc)]$. 
  \item \textit{Identity}: $\exists e\forall a (ea=ae=a)$. $e$ is called the \textit{identity} element of $U$.
  \item \textit{Enemies}: $\forall a\exists b (ab= ba = e)$. The element, $b$, that makes this property true for another element, $a$, is said to be the \textit{enemy} of $a$.
  \end{itemize}
  Note that commutativity need not hold, i.e., it is not necessary for all elements $a, b$ to have the property that $ab=ba$.

  Prove the following properties of $U$.
  \begin{parts}
  \item The identity element, $e$, is unique in $U$.
    \begin{solution}
      % Enter your solution here.
    \end{solution}
  \item Every element has a unique enemy.
    \begin{solution}
      % Enter your solution here.
    \end{solution}
  \item Given elements $a$ and $b$, the enemy of $ab$ is the same as the enemy of $b$ combined with the enemy of $a$.
    \begin{solution}
      % Enter your solution here.
    \end{solution}
  \end{parts}

\end{questions}

\end{document}